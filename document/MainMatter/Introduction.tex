\chapter*{Introducción}\label{chapter:introduction}
\addcontentsline{toc}{chapter}{Introducción}

En un mundo cada vez más interconectado, la seguridad de la información se ha convertido en una necesidad fundamental. La criptografía, como disciplina, juega 
un papel crucial en la protección de datos, proporcionando confidencialidad, integridad y autenticación en diversos sistemas tecnológicos. Desde la 
protección de comunicaciones personales hasta el aseguramiento de transacciones bancarias, sus aplicaciones son amplias y esenciales para el desarrollo de un entorno 
digital confiable. Dos de las tecnologías que fundamentan su seguridad son las \textbf{infraestructuras de clave pública} (\textit{PKI}) y la \textbf{\textit{blockchain}}.

Una de las principales ventajas de las \textit{PKI} es que proporcionan \textbf{autenticación fuerte}, ya que las claves privadas son únicas para cada 
entidad y solo estas pueden ser utilizadas para firmar o cifrar datos. Además, la \textbf{integridad de los datos} se asegura mediante el uso de firmas 
digitales, garantizando que los mensajes no sean alterados durante su transmisión. Otro beneficio clave de las \textit{PKI} es su capacidad para manejar grandes volúmenes 
de certificados y claves a través de un sistema organizado de autoridades certificadoras (\textit{CA}), lo que permite la gestión eficiente de la autenticación a gran escala.

La tecnología \textbf{\textit{blockchain}}, desde su surgimiento, también ha tenido un impacto enorme en el ámbito de la informática y las finanzas. Se caracteriza por 
su naturaleza \textbf{descentralizada}, \textbf{inmutable} y \textbf{transparente}, lo que la convierte en una solución ideal para aplicaciones que requieren registros 
seguros, auditables y verificables, como las transacciones financieras y la gestión de contratos inteligentes. \textit{Blockchain} permite a las partes interactuar de manera confiable 
sin necesidad de \textit{intermediarios}, lo cual reduce significativamente los costos y riesgos asociados a las transacciones digitales.

A pesar de representar enfoques distintos, se podría aprovechar lo mejor de ambas tecnologías a partir de la integración conjunta de las mismas. Una \textit{PKI} puede proporcionar una \textbf{autenticación robusta} 
y una \textbf{gestión eficiente de claves} para los nodos de la \textit{blockchain}, asegurando que solo los participantes legítimos puedan interactuar con la red. Aunque \textit{blockchain} asegura la integridad de los datos mediante su estructura 
descentralizada y su consenso distribuido, la identificación y verificación de los participantes de la red aún requieren mecanismos seguros, como los proporcionados por una \textit{PKI}. A través de este enfoque conjunto, la combinación de 
ambas tecnologías podría mejorar significativamente la confianza en los sistemas basados en \textit{blockchain}, especialmente en contextos donde la autenticidad de los participantes y la seguridad de las transacciones son críticos.

\section{Diseño Teórico}
\subsection*{Problema Científico}

Las redes blockchain requieren mecanismos seguros y confiables para la autenticación de nodos, evitando ataques de suplantación de identidad y mejorando la confianza en la red. Sin embargo, las soluciones tradicionales de autenticación suelen ser centralizadas o difíciles de integrar en sistemas distribuidos. 

\subsection*{Pregunta científica:} ¿Es posible implementar un mecanismo de validación confiable para los nodos de una red blockchain de Hyperledger Fabric utilizando una PKI externa?

\subsection*{Objeto de Estudio}
Infraestructura de clave pública aplicada a la redes blockchain para la autenticación segura de nodos.

\subsection*{Objetivo General}\
Diseñar e implementar una Autoridad Certificadora (CA) como mecanismo de autenticación segura de los nodos en la red Blockchain.

\subsection*{Objetivos específicos}\
\begin{enumerate}
    \item Analizar los esquemas de autenticación utilizados en redes .
    \item Diseñar una CA adaptada a las necesidades de autenticación de nodos en blockchain.
    \item Implementar la solución en un entorno de prueba utilizando herramientas de seguridad criptográfica.
    \item Evaluar el rendimiento y seguridad de la solución propuesta en comparación con otros métodos de autenticación.
\end{enumerate}

\subsection*{Campo de Acción}
La plataforma de Hyperledger Fabric

\subsection*{Hipótesis}
La implementación de una PKI en una red blockchain permitirá mejorar la seguridad y confiabilidad de la autenticación de nodos, reduciendo el riesgo de ataques de suplantación de identidad y fortaleciendo la integridad del ecosistema distribuido.

\section{Estructura del trabajo}

Esta tesis está organizada en tres capítulos principales, los cuales se describen a continuación:

\begin{itemize}
    \item \textbf{Capítulo 1: Preliminares} \\
    En este capítulo se aborda el marco teórico necesario para comprender los conceptos fundamentales sobre PKI y blockchain. El objetivo es proporcionar al lector el conocimiento necesario para entender los conceptos clave que se desarrollarán en el resto del trabajo.
    
    \item \textbf{Capítulo 2: Estado del Arte} \\
    Este capítulo presenta un informe sobre los trabajos más recientes relacionados con el tema de la tesis. Se analiza la literatura existente sobre PKI y blockchain, con especial énfasis en la integración de ambas.
    
    \item \textbf{Capítulo 3: Resultados y Discusión} \\
    En este capítulo se expone la solución propuesta, explicando en detalle cómo se lleva a cabo la implementación de la PKI para la autenticación segura de nodos en Hyperledger Fabric. Además, se realiza una discusión sobre los resultados obtenidos y su comparación con soluciones existentes.
\end{itemize}
