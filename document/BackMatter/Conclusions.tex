\begin{conclusions}
    \section{Conclusiones}

    En el presente trabajo se han presentado los conceptos fundamentales relacionados con las infraestructuras de clave pública (PKI) y las redes blockchain, destacando sus características principales y su relevancia tecnológica. Por un lado, la PKI provee los mecanismos necesarios para garantizar la identidad y confianza mediante el uso de certificados digitales, respaldados por entidades de confianza como las Autoridades Certificadoras (CAs). Por otro lado, la blockchain, con su descentralización, inmutabilidad, transparencia y seguridad, se erige como una solución poderosa para registrar y verificar transacciones de forma distribuida. 
    Tambien se ha explorado la sinergia entre estas dos áreas fundamentales mediante el análisis del estado del arte ha evidenciado propuestas innovadoras, como CertLedger y modelos de PKI descentralizada, que buscan superar las limitaciones de los sistemas tradicionales. Estas propuestas destacan ventajas en términos de transparencia, trazabilidad y eficiencia en la gestión y revocación de certificados digitales. En este sentido, la integración de una PKI con Hyperledger Fabric se presenta como una solución práctica y consolidada, capaz de aprovechar las fortalezas de ambas tecnologías.
    
    Entre las principales conclusiones se destacan:
    
    \begin{itemize}
        \item \textbf{Fusión de fundamentos teóricos y práctica innovadora:} La comprensión profunda de las propiedades de la blockchain, combinada con el análisis crítico del estado del arte, permitió diseñar una solución que integra una PKI tradicional con las capacidades avanzadas de Hyperledger Fabric.
        \item \textbf{Automatización y eficiencia en la gestión de identidades:} La implementación de una Autoridad Certificadora (CA) propia, desarrollada en Node.js, y la automatización del despliegue mediante archivos YAML y scripts en Bash, optimizan la incorporación de nodos y la administración de certificados, reduciendo la dependencia de configuraciones manuales.
        \item \textbf{Seguridad y transparencia reforzadas:} Al aprovechar las características inmutables y transparentes de la blockchain, se garantiza que los certificados digitales sean emitidos, revocados y auditados de manera segura, mejorando la autenticación de nodos mediante el Membership Service Provider (MSP) de Hyperledger Fabric y la implementación de TLS para comunicaciones seguras.
        \item \textbf{Escalabilidad y flexibilidad:} La solución presentada se adapta a diferentes escenarios y demandas operativas, permitiendo su implementación en redes blockchain permisionadas y ofreciendo la posibilidad de expandirse o integrarse con otras plataformas blockchain en el futuro.
    \end{itemize}
    
    En resumen, la combinación de una PKI con Hyperledger Fabric, respaldada por un sólido marco teórico y un análisis crítico del estado del arte, demuestra ser una estrategia prometedora para mejorar la autenticación y gestión de identidades en redes distribuidas.
    
\end{conclusions}


